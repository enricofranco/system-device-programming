\documentclass{report}
\usepackage[T1]{fontenc}
\usepackage[utf8]{inputenc}
\usepackage[english]{babel}
\usepackage{graphicx}
\usepackage[hidelinks]{hyperref}
\usepackage{enumitem}
\usepackage{fancyhdr}
\pagestyle{fancy}
\lhead{\textbf{System and device programming}}
\rhead{Laboratory 10}
\lfoot{Enrico Franco}
\rfoot{Politecnico di Torino}
\author{Enrico Franco}
\title{System and Device Programming \\
	Laboratory 10 - Exercise 2}
\begin{document}
\section*{Exercise 2}

First, it is needed to check that the number of arguments is correct, i.e.,\@ greater than 3. Otherwise the program will exit and return the value 1.

A semaphore for each reader thread is needed, in order to enable it to read an entry in the directory tree, initialized to one. A semaphore for the comparing thread is needed, in order to activate it when all threads have completed the reading of an entry, initialized to the number of threads.

In order to increase a little bit performance, recursion in vising directories is interrupted when a difference is found in some pathname. This is done by comparing thread, which sets a boolean flag for each reading thread when a single difference is discovered.
\end{document}