\documentclass{report}
\usepackage[T1]{fontenc}
\usepackage[utf8]{inputenc}
\usepackage[english]{babel}
\usepackage{fancyhdr}
\pagestyle{fancy}
\lhead{\textbf{System and device programming}}
\rhead{Laboratory 3}
\lfoot{Enrico Franco}
\rfoot{Politecnico di Torino}
\author{Enrico Franco}
\title{System and Device Programming \\
	Laboratory 3 - Exercise 2}
\begin{document}
\section*{Exercise 2}
First, it is needed to check that the number of arguments is correct, otherwise the program will exit and return the value 1.

The basic idea is to proceed in reading or writing if it is safe, i.e.\@ read if no writing operation is running and write if no reading or writing operation is running, otherwise wait for the ``turn''. In order to do this, it is needed to keep track of the number of readers and writers:
\begin{itemize}
\item \texttt{writers} contains the number of threads intended to write;
\item \texttt{writing} contains the number of threads ready to write;
\item \texttt{reading} contains the number of threads ready to read.
\end{itemize}
By using those variables, a reader needs to wait if a writer is intended to write and until all writing operations have been completed. In a similar way, a writer needs to wait until all reading and possibly writing operations have been completed.

After the execution of its own critical section, a thread needs to possibly wake up every other waiting thread, therefore a \texttt{cond\_broadcast} is used.
\end{document}