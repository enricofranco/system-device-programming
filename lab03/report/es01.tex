\documentclass{report}
\usepackage[T1]{fontenc}
\usepackage[utf8]{inputenc}
\usepackage[english]{babel}
\usepackage{fancyhdr}
\pagestyle{fancy}
\lhead{\textbf{System and device programming}}
\rhead{Laboratory 3}
\lfoot{Enrico Franco}
\rfoot{Politecnico di Torino}
\author{Enrico Franco}
\title{System and Device Programming \\
	Laboratory 3 - Exercise 1}
\begin{document}
\section*{Exercise 1}
First, it is needed to check that the number of arguments is correct, otherwise the program will exit and return the value 1.

Then, it is needed to check that the first argument represents a correct probability, i.e.\@ a number between 0 and 100, otherwise the program will exit and return the value 2.

This solution uses six semaphores in order to synchronize consumer and producer threads:
\begin{itemize}
\item \texttt{normalEmpty} and \texttt{normalFull} are used to manage the normal priority queue, therefore initialized to \texttt{BUFFER\_SIZE} and 0, respectively;
\item \texttt{urgentEmpty} and \texttt{urgentFull} are used to manage the high priority queue, therefore initialized to \texttt{BUFFER\_SIZE} and 0, respectively;
\item \texttt{globalEmpty} and \texttt{globalFull} are used to simply notify that an element has been produced or consumed respectively. Therefore they are initialized to \texttt{BUFFER\_SIZE * 2} and 0, respectively.
\end{itemize}
A data in the buffer is described by \texttt{typedef struct bufferData}. 

The producer waits on the \texttt{globalEmpty} semaphore before putting an element in the buffer and signals that it put an element on the \texttt{globalFull} semaphore. The consumer, viceversa, waits on the \texttt{globalFull} semaphore before extracting an element from a buffer and signals that an element has been extracted on the \texttt{globalEmpty} semaphore.

The producer produces the element by adding the timestamp and the buffer identifier according to the probability \texttt{p}, then, after waiting on the \texttt{globalEmpty} semaphore, it adds the element to the proper buffer, following the canonical \emph{Producer \& Consumer} scheme (wait on the \texttt{empty} semaphore and signal on the \texttt{full} one) and finally it signals on the \texttt{globalFull} semaphore.

The consumer, after waiting on \texttt{globalFull}, executes a \texttt{sem\_trywait} (non-blocking system call) on the \texttt{urgentFull} semaphore. If this call succeeds, the consumer consumes a data from the urgent buffer and it signals on the \texttt{urgentEmpty} semaphore, otherwise it waits on the \texttt{normalEmpty} semaphore, it consumea a data from the normal buffer and finally it signals on the \texttt{normalFull} semaphore.
\end{document}