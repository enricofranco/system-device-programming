\documentclass{report}
\usepackage[T1]{fontenc} % codifica dei font
\usepackage[utf8]{inputenc} % lettere accentate da tastiera
\usepackage[english]{babel} % lingua del documento
\usepackage{fancyhdr}
\pagestyle{fancy}
\lhead{Franco Enrico}
\chead{\textbf{System and device programming}} 
\rhead{Laboratory 1}
\author{Enrico Franco}
\title{System and Device Programming \\
	Laboratory 1 - Exercise 1}
\begin{document}
\section*{Exercise 1}
The exercise is quite simple and the algorithm is straightforward without particular features.

First, it is needed to check that the number of arguments is correct, otherwise the program will exit and return the value 1.

The directory \texttt{data/} is created using the system call \texttt{system} which launches the command \texttt{mkdir}. If an error occures during the execution of the command \texttt{mkdir} the program will exit and return the value 2.

At this point it is possible to create \texttt{n} files as specified in the text of the exercise.
For \texttt{n} times it is needed to:
\begin{enumerate}
\item Generate the filename;
\item Open the file or create it if it does not exist;
\item Generate a random number \texttt{k};
\item Write \texttt{k} random numbers in the file;
\item Close the file.
\end{enumerate}
\end{document}