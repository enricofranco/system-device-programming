\documentclass{report}
\usepackage[T1]{fontenc}
\usepackage[utf8]{inputenc}
\usepackage[english]{babel}
\usepackage{fancyhdr}
\pagestyle{fancy}
\lhead{\textbf{System and device programming}}
\rhead{Laboratory 4}
\lfoot{Enrico Franco}
\rfoot{Politecnico di Torino}
\author{Enrico Franco}
\title{System and Device Programming \\
	Laboratory 4 - Exercise 2}
\begin{document}
\section*{Exercise 2}
First, it is needed to check that the number of arguments is correct, otherwise the program will exit and return the value 1.

Then, it is needed to check that the second argument represents a correct size for an array or a matrix, i.e.\@ a strict positive number, otherwise the program will exit and return the value 2.

After the allocation of \texttt{v1}, \texttt{v2} and \texttt{mat}, the main program creates \texttt{k} threads, passing to them:
\begin{itemize}
\item \texttt{k}, i.e.\@ size of the arrays \texttt{v1}, \texttt{v2} and the square matrix \texttt{mat};
\item pointers to \texttt{v1}, \texttt{v2} and \texttt{mat};
\item pointer \texttt{intermediateArray}, pointing to an intermediate array in which store the intermediate product \texttt{mat * v2};
\item \texttt{index}, indicating on which of the matrix and intermediate array the thread has to work.
\end{itemize}
and waits for their termination. Then, it simply print the result;

Each thread performs the scalar product between the \texttt{index}-th row of the matrix \texttt{mat} and the array \texttt{v2}, stores the result in the array \texttt{intermediateArray} in position \texttt{index} and, in mutual exclusion, increments a global counter in order to exploit a synchronization barrier paradigm. In this way, the last thread is able to recognize such a situation and it executes the scalar product between \texttt{v1} and the intermediate array, producing the desired result in a global variable seen by the main program.
\end{document}