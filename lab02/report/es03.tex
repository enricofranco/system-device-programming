\documentclass{report}
\usepackage[T1]{fontenc}
\usepackage[utf8]{inputenc}
\usepackage[english]{babel}
\usepackage{fancyhdr}
\pagestyle{fancy}
\lhead{\textbf{System and device programming}}
\rhead{Laboratory 2}
\lfoot{Enrico Franco}
\rfoot{Politecnico di Torino}
\author{Enrico Franco}
\title{System and Device Programming \\
	Laboratory 1 - Exercise 3}
\begin{document}
\section*{Exercise 3}
First, it is needed to check that the number of arguments is correct, otherwise the program will exit and return the value 1.

The main thread generates two random arrays and stores them in two binary files according to the specifications (using the system calls \texttt{open} and \texttt{write}) and allocates three semaphores:
\begin{itemize}
\item \texttt{mutexRead} protects the critical section, because the variable \texttt{g}, where the content of files is read, is global (i.e.\@ shared among threads). This semaphore is initialized to 1 to permit the access to the critical section to first thread;
\item \texttt{processing} used by the threads, inside the critical section, to inform the server that a new value is available into the variable \texttt{g}. This semaphore is initialized to 0 in order to block the server;
\item \texttt{dataProcessed} used by the server, inside its loop, to inform the thread in the critical section that the data has been elaborated and the result is available into the variable \texttt{g}. This semaphore is initialized to 0 in order to block the thread in the critical section before printing the result.
\end{itemize}

In order to manage the termination, the clients will write value -1 into the variable \texttt{g} when the reading of the file is terminated so that the server can recognize such a situation. The server will terminate its execution after receiving the value -1 two (number of threads) times.
\end{document}