\documentclass{report}
\usepackage[T1]{fontenc}
\usepackage[utf8]{inputenc}
\usepackage[english]{babel}
\usepackage{fancyhdr}
\pagestyle{fancy}
\lhead{\textbf{System and device programming}}
\rhead{Laboratory 2}
\lfoot{Enrico Franco}
\rfoot{Politecnico di Torino}
\author{Enrico Franco}
\title{System and Device Programming \\
	Laboratory 2 - Exercise 1}
\begin{document}
\section*{Exercise 1}
First, it is needed to check that the number of arguments is correct, otherwise the program will exit and return the value 1.

The file specified on the command line must be read line by line by the parent process, so  it is needed to use \texttt{fopen} to open the file and \texttt{fgets} to read it line by line.

The parent process (1) forks and executes different operations between itself and its child:
\begin{itemize}
\item The parent process (1) allocates two signal handlers which execute function \texttt{signalHandlerParent} for signals \texttt{SIGUSR1} and \texttt{SIGUSR2} and start reading the file, line by line, until the global variable \texttt{TERMINATE} is set to \texttt{FALSE}, rewinding the file when it terminates its reading.

Before printing the read line, the parent tests the global variable \texttt{PRINT\_LINE} and it prints the read line only if the variable is set to \texttt{TRUE};
\item The child process (2) allocates a signal handler, different from the previous one, which executes function \texttt{signalHandlerChild} for signal \texttt{SIGUSR2}, then it forks again:
\begin{itemize}
\item It (2) loops until the global variable \texttt{TERMINATE} is set to \texttt{FALSE} sending a signal \texttt{SIGUSR1} to the main process (1) at random intervals;
\item Its child (3) sleeps 60 seconds, then it sends a signal to its parent (2), which will cause its termination and, in cascade, the termination of the main process.
\end{itemize}
\end{itemize}

\paragraph{\texttt{signalHandlerChild}}
Child process (2) has only to manage signal \texttt{SIGUSR2}, which will cause, in cascade, the termination of the entire program. To obtain this behavior, the handler sets the global variable \texttt{TERMINATE} to \texttt{TRUE}.

\paragraph{\texttt{signalHandlerParent}}
Parent process (1) has to manage both \texttt{SIGUSR1} and \texttt{SIGUSR2}.

The first one will cause the inversion of the variable \texttt{PRINT\_LINE} which is obtained by a bitwise XOR with \texttt{TRUE}.

The second one will cause the termination of the entire program. To obtain this behavior, the handler sets the global variable \texttt{TERMINATE} to \texttt{TRUE}.

\end{document}