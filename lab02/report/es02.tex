\documentclass{report}
\usepackage[T1]{fontenc}
\usepackage[utf8]{inputenc}
\usepackage[english]{babel}
\usepackage{fancyhdr}
\pagestyle{fancy}
\lhead{\textbf{System and device programming}}
\rhead{Laboratory 2}
\lfoot{Enrico Franco}
\rfoot{Politecnico di Torino}
\author{Enrico Franco}
\title{System and Device Programming \\
	Laboratory 2 - Exercise 2}
\begin{document}
\section*{Exercise 2}
First, it is needed to check that the number of arguments is correct, otherwise the program will exit and return the value 1.

The maximum tree height is received from the command line and it is stored on the global variable \texttt{n}.

In order to properly stop the thread tree generation and to print the correct sequence of thread identifiers, each thread must know its level and its previous threads identifiers. The \texttt{struct thread\_t} stores an array of \texttt{pthread\_t} which identifies the sequence of threads and an \texttt{int} which identifies the thread level.

Each thread tests its level:
\begin{itemize}
\item If it is lower than \texttt{n}, it copies in a new location the identifiers of previous threads, adds its own identifier and increments the level.
\item If it is equal to \texttt{n}, it means that it is a leaf of the thread tree and the thread simply prints its identifier followed by the identifiers of previous threads.
\end{itemize}
\end{document}