\documentclass{report}
\usepackage[T1]{fontenc}
\usepackage[utf8]{inputenc}
\usepackage[english]{babel}
\usepackage{graphicx}
\usepackage[hidelinks]{hyperref}
\usepackage{fancyhdr}
\pagestyle{fancy}
\lhead{\textbf{System and device programming}}
\rhead{Laboratory 8}
\lfoot{Enrico Franco}
\rfoot{Politecnico di Torino}
\author{Enrico Franco}
\title{System and Device Programming \\
	Laboratory 8 - Exercise 2}
\begin{document}
\section*{Exercise 2}

First, it is needed to check that the number of arguments is correct, otherwise
the program will exit and return the value 1.

Defining \texttt{UNICODE} and including \texttt{tchar}, it is possible to use \texttt{\_T} macro and functions which will be automatically converted to the UNICODE versions.

Input file must be read using \texttt{\_ftscanf} because it is an ASCII file, i.e.\@ it must be interpreted as text, while output file must be written using Windows \texttt{WriteFile}, i.e.\@ binary serialized file.

At the end of the copy, the generated file is read again, using Windows \texttt{ReadFile} this time and printing read data on screen to ensure that data has been copied correctly.

Errors on \texttt{\_tfopen}, used to allocate the file pointer on input file, \texttt{CreateFile}, used to create handles on output file, and the number of correctly written bytes are checked and they will cause the termination of the process, returning a recognizable error code, i.e.\@ different from the ``canonical'' zero.

\end{document}